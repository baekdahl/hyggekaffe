%\textbf{Problem Statement}\\
%How can we, with a standard inexpensive webcam, correctly detect a pool table and identify pool balls in mixed lighting?

%Intro
We have tried to develop a system that would fulfil the problem statement. Unfortunately some of the key elements of the system does not work as stated in the requirement specification. This is partly due to the input data from the webcam, but also because of the solutions made.

%Detecing table
The detection of the table works according to the requirements specification. The pool table is detected when the table takes up 75\% of the image. The detector has been tested with the table having many different angles compared to the image frame, and the detection provides a cropped rotated image of the table. The detection can also function in some situations where the table is occluded, even though this is not required.

%Method for detecting states
We have developed a method that will identify different states of the pool game. The main part of this is detection of human interaction. The detection of human interaction works satisfactorily, and successfully prevents the ball detection from detection foreign objects as balls. The interaction detection did not work for small interactions, such as a cue stick placed on the edge of the cloth, but there were no problems for interactions observed during normal gameplay.

%Finding position of balll
The positions of balls were found with high accuracy when the balls were laying apart. For balls in clusters, and especially if the clusters contained the same type of colors, the position will sometimes be detected incorrectly. This is caused by the inability to separate the different colors contained in the balls. The reason for this could be found in the developed method, which only considers the hue variance when detecting balls. This project does however partially succeed in the detection of balls in clusters, as opposed to other projects like \cite{supportBilliard}.

%Identifying balls
The identification of balls works in the optimal environment, i.e. good illumination and balls showing a minimal white area. The color classification is based on the Euclidean distance, which makes it dependant on the illumination and thereby unable to work when lighting changes.
The system has problems classifying the striped balls when they show a minimum of color, and further more some colors are classified as each other. The reason for identification problems could be problems in the developed method, but there were also problems with the color quality from the used webcam. The webcam was not able to output images or videos with well separated colors. Often the orange, red and brown as well as the purple, blue and black had distributions very close to each other.
%GUI

A working prototype with a GUI was made. Although the GUI could be improved it is very good for illustrating the proof-of-concept.