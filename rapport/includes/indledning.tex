%Næsten alle firmaer og privatpersoner har idag en form for telefonsvarer. I de fleste tilfælde er det tilstrækkeligt med en voicemail, placeret centralt hos teleudbyderen, men for nogle er det praktisk at have en telefonsvarer stående lokalt.
%Hvis der f.eks. er brug for et flerbrugersystem, vidrestilling ud fra en række valgmuligheder, eller lignende.
%
%Telefonsvareren blev opfundet i 1935, så det er et produkt der har været på markedet meget længe. De tidlige telefonsvarere optog beskederne på magnetbånd, mens de moderne udgaver typisk anvender digitalt solid-state lager. Med moderne teknologi, kan en digital telefonsvarer laves meget billigt, og det er nemt at implementere nye funktioner. Endvidere gør det digitale lager det nemt at overføre beskeder til en PC, hvorfra beskeden f.eks. kan aflyttes via internettet. \cite{wiki:answering_machine}
%
%I denne rapport dokumenteres opbygningen af en digital telefonsvarer med mulighed for at gemme beskederne på eksternt medie og mulighed for opkobling til en PC.
%Da telefonsvareren er et gammelt produkt, og der længe har været efterspørgsel for denne type systemer, findes der allerede en del produkter på markedet som udfylder behovet, og produktet er derfor ikke nyskabende, men kan ses som et alternativ til de eksisterende løsninger.
%
%Produktets målgruppe er små virksomheder, som har brug for at medarbejderne nemt kan aflytte beskeder, rettet specifikt til dem.
%
