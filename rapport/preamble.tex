\newcommand{\accel}[0]{\ensuremath{\tfrac{\text{m}}{\text{s}^2}}\text{ }}
\newcommand{\speed}[0]{\ensuremath{\tfrac{\text{m}}{\text{s}}}\text{ }}
\newcommand{\te}[1]{\text{#1}}
\newcommand{\citep}[1]{\cite{#1}}
\newcommand{\projekt}{PoolTracker}
\newcommand{\projektemne}{}
\newcommand{\gruppe}{11gr822}
\newcommand{\gruppemedlemmer}{Jesper Bækdahl og Simon Have}
\newcommand{\vejleder}{}
\newcommand{\Ohm}{$\Omega${\ }}
\newcommand{\ohm}{$\Omega${\ }}
\newcommand{\bet}{$\beta$}
\newcommand{\Bet}{$\beta$}
\newcommand{\degree}{\ensuremath{^\circ}}
\newcommand{\HRule}{\rule{\linewidth}{0.5mm}}


\documentclass[pdftex,10pt,english,a4paper,twoside,openright,
draft,
%final,
]{memoir}

%
% Packages
%
% Varioref, dansk
\usepackage[english]{varioref}

\usepackage[notref,notcite,color,
final
]{showkeys}


%
% Settings for varioref and showkeys
%
\makeatletter
\AtBeginDocument{
% reinstate \vref
  \DeclareRobustCommand\vref{\@ifstar
	{\let\vref@space\relax\vr@f}%
	{\let\vref@space\nobreakspace\vr@f}}
  \@ifpackagewith{showkeys}{notref}{%
  % for the notref option:
	\def\vr@f#1{%
	 \leavevmode\unskip\vref@space
	 \ref{#1}% added next line:
	 \let\label\SK@label
	 \vpageref[\unskip]{#1}}}%
  {% for the normal case
  \def\vr@f#1{%
	\leavevmode\unskip\vref@space
	\ref{#1}% added next line:
	\let\label\SK@label\let\ref\SK@ref\let\pageref\SK@pageref
	\vpageref[\unskip]{#1}}%
  }
}
\renewcommand\paragraph{%
   \@startsection{paragraph}{4}{0mm}%
      {-\baselineskip}%
      {.5\baselineskip}%
      {\normalfont\normalsize\bfseries}
}

\usepackage{pdfsync}
\usepackage{lmodern}

\usepackage{url}

% Turns references into links
\usepackage[
			colorlinks=false,
			breaklinks,
			unicode=true
			pdfduplex=DuplexFlipLongEdge,
			pdfborder={1 0 1},
			pdftitle={\projekt},
			pdfauthor={\gruppe,
				VGIS, 8. semester,
				SICT,
				Aalborg Universitet},
			pdfsubject={\projektemne},
			pdfkeywords={\gruppe,
				LaTeX,
				domo,arigatou},
			plainpages=false,
			%pdftex,
			final,
			]{hyperref}

\usepackage{hypcap}
\usepackage{multirow}

\usepackage{memhfixc}

% Input encoding
\usepackage[utf8]{inputenc}

% Enables line breaks in URLs
\usepackage{breakurl}

% Graphics package
\usepackage[pdftex,final]{graphicx}
\DeclareGraphicsExtensions{.pdf,.png,.jpg} % Prioritized list of file endings
\graphicspath{{./images/}{./doxygen/}} % Sets default path for images

% Enables rotation of text
\usepackage{rotating}

%% American Math Society, avanceret matematikpakke
\usepackage{amsmath}
\usepackage{amsfonts}
\usepackage{amssymb}
\usepackage{mathrsfs}

% Gør det muligt at bruge \uline{} og \uuline{} til underlining og dobbelt underlinuing af math.
\usepackage{ulem}

% Enables the usage of columns
\usepackage{multicol}

% Fixme
\usepackage{fixme}
\fxsetup{layout=footnote}


% Sets the writing language
\usepackage[english]{babel}
% Floats
\usepackage{morefloats}

%% Enables wrapping text around figures
%\usepackage{wrapfig}

%\usepackage[rflt]{floatflt}

% Enables custom spacing
\DisemulatePackage{setspace}
\usepackage{setspace}

% Adds the possibility of using if/then in LaTeX
\usepackage{ifthen}

% Colors
\usepackage{color}
	%\definecolor{sourceYellow}{rgb}{1,1,0.85}
	\definecolor{codecomment}{rgb}{0.5,0,0.5} % 4b5fbf
	\definecolor{stringsGreen}{rgb}{0,0.5,0}
	\definecolor{keywordsRed}{rgb}{0.6,0,0} 
	\definecolor{commentsRed}{rgb}{0.8,0,0} 
	\definecolor{keywordsBlue}{rgb}{0.0,0.0,0.84}
	\definecolor{codedefine}{rgb}{0,0.5,0.5} % 007f7f

% Listings
% XXX - change to listingsutf8 in near future
\usepackage[final]{listings}
	\renewcommand{\lstlistingname}{Code}
	\renewcommand{\lstlistlistingname}{Code}
	\lstset{ %
	language=C,							% choose the default language
	basicstyle=\ttfamily\footnotesize,	% the style of the fonts that are used for the code
	stringstyle=\color{stringsGreen},	% style of strings
	commentstyle=\color{commentsRed},	% comments
	backgroundcolor=\color{white},		% choose the background color. You must add \usepackage{color}
	keywordstyle={\bfseries\color{keywordsBlue}},
	showstringspaces=false,				% [bool] underline spaces within strings
	numbers=left,						% [left/right] where to put the line-numbers
	numberstyle=\tiny,					% the size of the fonts that are used for the line-numbers
	stepnumber=1,						% [int] the step between two line-numbers. If it's 1 each line will be numbered
	numbersep=5pt,						% [length] how far the line-numbers are from the code
	showspaces=false,					% [bool] show spaces within strings adding particular underscores
	showtabs=false,						% [bool] show tabs within strings adding particular underscores
	frame=single,						% adds a frame around the code
	frameround=tttf,					% rounding of frames
	tabsize=2,							% [int] sets default tabsize to 2 spaces
	captionpos=b,						% [b|t] sets the caption-position to bottom
	breaklines=true,					% [bool] sets automatic line breaking
	breakatwhitespace=false,			% [bool] sets if automatic breaks should only happen at whitespace
	inputencoding=utf8,					% default input encoding
	escapeinside={*@}{@*},				% if you want to use LaTeX within your code, use *@ and @* to begin and end the code
	directives={define,elif,else,endif,error,if,ifdef,ifndef,line,include,pragma,undef,warning},
	%emph={define},emphstyle={\color{codedefine}}
	}
%	\lstset{emph={asm,auto,bool,break,case,catch,char,class,const,const_cast,continue,default,delete,do,double,dynamic_cast,else,enum,explicit,export,extern,false,float,for,friend,goto,if,inline,int,long,mutable,namespace,new,operator,private,protected,public,register,reinterpret_cast,return,short,signed,sizeof,static,static_cast,struct,switch,template,this,throw,true,try,typedef,typeid,typename,union,unsigned,usin,void},
%	emphstyle=\color{blue}\bfseries,
%	emph={[2]memcpy,floor,ISR},
%	emphstyle={[2]\color{keywordsRed}},
%	}
%\lstset{emph={define},emphstyle=\color{stringsGreen}}


\usepackage{lastpage}

\newcommand{\xxtiny}{\fontsize{2}{2}\selectfont}

%
% Margins
%
\setlrmarginsandblock{*}{3.5cm}{0.75} % højre og venstre
\setulmarginsandblock{3cm}{*}{1.2}	% top og bund
\checkandfixthelayout[nearest]		% specifikt valg af højde algoritme

%
% Headings
%
% Change normal headers and footers
\makeoddhead{headings}{}{}{\small\rightmark}
\makeevenhead{headings}{\small\leftmark}{}{}

\makeoddfoot{headings}{}{}{\small\thepage}
\makeevenfoot{headings}{\small\thepage}{}{}

\makeheadrule{headings}{\textwidth}{\normalrulethickness}
\makefootrule{headings}{\textwidth}{\normalrulethickness}{\footruleskip}

% Change chapter pages
\copypagestyle{chapter}{plain}
\makeoddfoot{chapter}{}{}{\small\thepage}
\makeevenfoot{chapter}{\small\thepage}{}{}
\makefootrule{chapter}{\textwidth}{\normalrulethickness}{\footruleskip}

%
% Section titles
%
\setsecnumdepth{subsection}
\maxsecnumdepth{subsection}
\setsecheadstyle{\Large\bfseries\sffamily\raggedright}
\setsubsecheadstyle{\large\bfseries\sffamily\raggedright}
\setsubsubsecheadstyle{\normalsize\bfseries\sffamily\raggedright}
\raggedbottomsectiontrue

%
% Table of Contents
%
\renewcommand{\contentsname}{Table of contents}
%\renewcommand{\tocname}{Indholdsfortegnelse}
\settocdepth{section}

% Change spacing in ToC
\makeatletter
\renewcommand*\l@section{\@dottedtocline{1}{1.5em}{2.8em}}
\renewcommand*\l@subsection{\@dottedtocline{2}{3.5em}{2.4em}}
\renewcommand*\l@subsubsection{\@dottedtocline{3}{4.3em}{3.2em}}
\makeatother

%
% Chapter
%
\renewcommand\chapnamefont{\huge\bfseries\sffamily}
\renewcommand\chapnumfont{\chapnamefont}
\renewcommand\chaptitlefont{\Huge\bfseries\sffamily\raggedright}

\usepackage{calc}
\makeatletter
\setlength\midchapskip{0pt}

% Define a new chapter style
\makechapterstyle{10gr621}{
  \newcommand{\chapterrule}{\rule[.3\baselineskip]{\textwidth}{1pt}}
  \renewcommand\chapnamefont{\Large\sffamily}
  \renewcommand\chapnumfont{\Large\sffamily\centering}
  \renewcommand\chaptitlefont{\huge\bfseries\sffamily\centering}
  \renewcommand\printchaptertitle[1]{%
	\chaptitlefont
	\ifdim\@tempdimc > 0pt\relax % one line
	 \chapterrule \\
	 ##1
	 \chapterrule
	\else % two+ lines
		>{\chaptitlefont\arraybackslash}p{\textwidth-2\tabcolsep}
	 \chapterrule \\
	 ##1
	 \chapterrule
	\fi
  }
}
\makeatother
\chapterstyle{10gr621}

%
% Figure Captions
%
\captionnamefont{\bfseries\sffamily\small}
\captiontitlefont{\small}
\changecaptionwidth
\captionwidth{.8\textwidth}
\precaption{\vspace{\baselineskip}}
\hangcaption

%
% Bibliography management
%
\bibliographystyle{plain}

%
% PDF information
%
\pdfinfo{
   /Author (Jesper Bækdahl, Rolf R. Madsen og Simon Hartmann Have)
   /Title  (Quality assessment of metal objects using
computer vision.
)
   %/CreationDate (\date)
   /Subject (Computer Vision)
   /Keywords ()
}


%%% AUTOREF
%% Creates a new command, aref, that is essentially the same as autoref, but includes page number if label is on another page.
\newcommand{\aref}[1]{\autoref{#1}}%\ifthenelse{\equal{\pageref{#1}}{\thepage}}{}{, side \pageref{#1}}}

%% Environment for writing hex numbers
\newcommand{\hex}[1]{\texttt{#1}}

%% Environment for writing function calls
\newcommand{\function}[1]{\texttt{\textbf{#1}}}

%% Environment for writing function calls
\newcommand{\pin}[1]{\texttt{\textit{#1}}}

%% Environment for writing function calls
\newcommand{\register}[1]{\textit{#1}}

%% Custom equation environment
\newenvironment{eqn}
{\begin{figure}[htp]\capstart\begin{center}} %[XXX - Skal ændres tilbage før aflevering - RJ]
%{\begin{figure}[!h]\begin{center}}
{\end{center}\end{figure}}
%{\begin{equation}}
%{\end{equation}}
%\newcommand{\pref}[1]{(\ref{#1})}

\usepackage{subfig}

\setlength{\parindent}{0pt}

\setlength{\itemsep}{1pt}

\setlength{\parskip}{.5ex}

% Making it possible to use colors and defining some new colors
\usepackage{color, colortbl}
\definecolor{purple}{rgb}{0.22,0.01,0.36}
\definecolor{orange}{rgb}{0.98,0.42,0.04}
\definecolor{brown}{rgb}{0.36,0.01,0.01}
\definecolor{gray}{gray}{0.9}
%56 3 93