\begin{figure}[H]
\centering
\subfloat
{
	\includegraphics[width=0.3\textwidth]{images/ballhist/0}
}
\subfloat
{
	\includegraphics[width=0.3\textwidth]{images/ballhist/8}
}
\caption{Color histogram of cue-ball and 8-ball}
\label{fig:ballhist-cue-8}
\end{figure}
Figure \ref{fig:ballhist-cue-8} shows the histograms of the cue-ball and the 8-ball. The cue-ball distribution is isolated in a low-saturation area, having a yellow hue around 50. The saturation of the other balls is generally above the white saturation, making it possible to identify white pixels by setting a saturation threshold.

The hue-saturation distribution of the 8-ball is scattered all over the range. The reason for this, is that black is undefined in hue-saturation space. Black will have to be detected by the brightness value, which is significantly lower than other balls. \fxnote{VALUE HISTOGRAM HERE.}

\begin{figure}[H]
\centering
\subfloat
{
	\includegraphics[width=0.3\textwidth]{images/ballhist/3}
}
\subfloat
{
	\includegraphics[width=0.3\textwidth]{images/ballhist/5}
}

\subfloat
{
	\includegraphics[width=0.3\textwidth]{images/ballhist/7}
}

\caption{Color histogram of balls: 3, 5 and 7}
\label{fig:ballhist-3-7}
\end{figure} 
Figure \ref{fig:ballhist-3-7} shows a situation where the balls are going to be difficult to separate. Depending on the lighting and camera settings, the color of 3, 5 and 7 have almost the same hue, and are only separable in saturation.

The histograms in figure \ref{fig:ballhist-3-7} also show one of the weaknesses of using HSB colorspace which is the that hue is defined an angular value that wraps between minimum and maximum. The consequence of this is that the three  balls which has red as their dominant hue have one distribution near minimum hue and one near maximum.
\begin{figure}[H]
\centering
\subfloat
{
	\includegraphics[width=0.3\textwidth]{images/ballhist/2}
}
\subfloat
{
	\includegraphics[width=0.3\textwidth]{images/ballhist/4}
}
\caption{Color histogram of balls: 2 and 4}
\label{fig:ballhist-2-4}
\end{figure}
Figure \ref{fig:ballhist-2-4} shows that the blue and purple balls are also challenging to separate. This is again because of the colors being almost equal to each other when the brightness in HSB is not used.

\begin{figure}[H]
\centering
\subfloat
{
	\includegraphics[width=0.3\textwidth]{images/ballhist/1}
}
\subfloat
{
	\includegraphics[width=0.3\textwidth]{images/ballhist/6}
}
\caption{Color histogram of balls: 1 and 6}
\label{fig:ballhist-1-6}
\end{figure} 
Figure \ref{fig:ballhist-1-6} shows the two balls which do not have close neighbors. The green ball does however contain colors that are very similar to the color of the table cloth, making it harder to separate from the background than the rest.

