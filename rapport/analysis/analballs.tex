To find the table and balls it is important to understand how the table and balls appear in different color spaces. In this section the colors of the table and the pool balls will be measured for further analysis.\\

\subsubsection{The table}
As written in the rules\ref{sec:rules} the table cloth colour has to be yellow-green, blue-green or electric blue. The cloth color should be constant, but dependent on the lightning the variance of the cloth on the input image could be significant.

The table which has been used for measuring is a pool table with blue-green cloth which is located in Aalborg University's lab in room A6-314!

Three different images of the pool table and one image with the cloth cut out are analysed first in HSV color space and then i RGB space. These analysis will be used in the solution chapter.

\begin{figure}[H]
\begin{center}
\leavevmode
\includegraphics[width=1\textwidth]{images/hsv_hist_table}
\end{center}
\caption{HSV: Image 1,2 and 3 are mixed lightning where 4 is the cutout of the cloth from image 1. All axis are aligned.}
\label{fig:tablehsv}
\end{figure}

\begin{figure}[H]
\begin{center}
\leavevmode
\includegraphics[width=1\textwidth]{images/rgb_hist_table}
\end{center}
\caption{RGB: Image 1,2 and 3 are mixed lightning where 4 is the cutout of the cloth from image 1. All axis are aligned.}
\label{fig:tablergb}
\end{figure}

\subsubsection{The balls}
The success of the system depends on the identifier to be able to separate the pool balls from each other. To analyze the separability, histograms of all the different solid balls has been computed. This has been done in hue-saturation space, to see if it is possible to be independent of brightness. Being independent of brightness will make the system more resistant towards changes in light. The histograms are grouped together based on the color similarity, to visualize how the colors can be separated

\begin{figure}[H]
\centering
\subfloat
{
	\includegraphics[width=0.05\textwidth]{images/balls/0.png}
}
\hspace{0.4\textwidth}
\subfloat
{
	\includegraphics[width=0.05\textwidth]{images/balls/8.png}
}

\subfloat
{
	\includegraphics[width=0.4\textwidth]{images/ballhist/0}
}
\subfloat
{
	\includegraphics[width=0.4\textwidth]{images/ballhist/8}
}
\caption{Color histogram of cue and 8}
\label{fig:ballhist-cue-8}
\end{figure}
Figure \ref{fig:ballhist-cue-8} shows the histograms of the cue-ball and the 8-ball. The cue-ball distribution is isolated in a low-saturation area, having a yellow hue around 50. The saturation of the other balls is generally above the white saturation, making it possible to identify white pixels by setting a saturation threshold.

The hue-saturation distribution of the 8-ball is scattered all over the range. The reason for this, is that black is undefined in hue-saturation space. Black will have to be detected by the brightness value, which is significantly lower than other balls. \fxnote{VALUE HISTOGRAM HERE.}

\begin{figure}[H]
\centering
\subfloat
{
	\includegraphics[width=0.05\textwidth]{images/balls/3.png}
}
\hspace{0.4\textwidth}
\subfloat
{
	\includegraphics[width=0.05\textwidth]{images/balls/7.png}
}

\subfloat
{
	\includegraphics[width=0.4\textwidth]{images/ballhist/3}
}
\subfloat
{
	\includegraphics[width=0.4\textwidth]{images/ballhist/7}
}
\caption{Color histogram of 3 and 7}
\label{fig:ballhist-3-7}
\end{figure} 
Figure \ref{fig:ballhist-3-7} shows a situation where the balls are going to be difficult to separate. Depending on the lighting and camera settings, the color of 3, 5 and 7 have almost the same hue, and are only separable in saturation.

\begin{figure}[H]
\centering
\subfloat
{
	\includegraphics[width=0.05\textwidth]{images/balls/1.png}
}
\hspace{0.4\textwidth}
\subfloat
{
	\includegraphics[width=0.05\textwidth]{images/balls/5.png}
}

\subfloat
{
	\includegraphics[width=0.4\textwidth]{images/ballhist/1}
}
\subfloat
{
	\includegraphics[width=0.4\textwidth]{images/ballhist/5}
}
\caption{Color histogram of 1 and 5}
\label{fig:wrongcamera}
\end{figure} 


\begin{figure}[H]
\centering
\subfloat
{
	\includegraphics[width=0.05\textwidth]{images/balls/2.png}
}
\hspace{0.4\textwidth}
\subfloat
{
	\includegraphics[width=0.05\textwidth]{images/balls/4.png}
}

\subfloat
{
	\includegraphics[width=0.4\textwidth]{images/ballhist/2}
}
\subfloat
{
	\includegraphics[width=0.4\textwidth]{images/ballhist/4}
}
\caption{Color histogram of 2 and 4}
\label{fig:wrongcamera}
\end{figure}

\textbf{Ball color composition}\\
To be able to detect whether or not a ball is solid or striped it is important to know how much white is in a ball. The images used can be seen in figure \ref{fig:ballscompo} which is based on the balls used in this project.

\begin{figure}[H]
\centering
\subfloat[Solid ball with number turned up.]
{
	\includegraphics[width=0.2\textwidth]{images/fy.png}
}
\subfloat[Striped ball with minimum white.]
{
	\includegraphics[width=0.2\textwidth]{images/fymax.png}
}
\subfloat[Striped ball with maximum white.]
{
	\includegraphics[width=0.2	\textwidth]{images/fymin.png}
}

\caption{Ball color compositions.}
\label{fig:ballscompo}
\end{figure}

In table \ref{fig:ballscompotable} the percentage of white pixels can be seen.\\


\begin{table}[H]
\centering
\begin{tabular}{|c|c|c|c|}
	\hline Figure & a) & b) & c) \\ 
	\hline Color & 583217 & 548839 & 175858 \\ 
	\hline White & 137359 & 173401 & 542900 \\ 
	\hline Percentage & 19.0\% & 24.0\% & 75.5\% \\ 
	\hline
\end{tabular}
\caption{Table with ball color composition.}
\label{fig:ballscompotable}
\end{table}

