There are several possibilities for making the system work with greater accuracy. Most importantly with a better camera, the detection and identification would be improved. The methods used for detection and identification could also be improved, or replaced by more sophisticated algorithms. When the detection and identification is good enough, the game rules can then be implemented and it will be possible to keep track of scores and game states.

The camera position could be changed from the current fixed location over the table. This would enable the system to be used from e.g. a handheld device like a smartphone. This would make the system more modern and versatile.

The system could also be extended by adding an extra camera. This camera could provide an alternative view of the balls. By positioning the two cameras optimally it could be avoided e.g. identifying a striped ball as the cue, because of the white part facing directly into the camera. Also reflections from a ball into a camera would be easier handled by having two cameras, since one of these probably would not receive the reflection.

Furthermore tracking of the cue stick would also make it possible to predict where the player is going to place his shot. Combining this with prediction of speed and trajectory, the system could be used for visualising how different ways of shooting will affect the game. This could be combined with augmented reality, where guidelines are projected to the table by a projector as in \cite{larsbopool} which currently only works with two balls.

For complete tracking the balls could be detected in all frames. By utilizing all frames, and developing a method that is able to track the balls when they are in motion, it will be possible to detect events such as which pockets the balls have been shot into.